\subsection{Sets}

\begin{itemize}
    \item A set is an unordered collection of objects. Two sets are equal if all elements are the same, i.e. ($A \subseteq B) \wedge (A \supseteq B)$.
    
    \item To denote an item being in a set, use $\in$. Example: $a \in \{a, b, c\}$.
    
    \item Important note: Sets can be a part of sets. Example: $A = \{a, \{b, c\}, d\}$, where $b \notin A$.
    
    \item A set that includes all even natural numbers can be denoted as $A = \{x \in \mathbb{N} : \exists n \in \mathbb{N} (x = 2n) \}$. This translates to "A is the set of all $x$ in natural numbers such that there exists $n$ in the set of all natural numbers where $x = 2n$". The format used in this is called set builder notation.
    
    \item Some important operators/definitions:
        \begin{itemize}
            \item $\emptyset$ is the empty set.
            \item $\mathbb{R}$ is the set of real numbers.
            \item $\mathbb{Q}$ is the set of rational numbers.
            \item $\wp (A)$ is the powerset of A (a set of all subsets of A, has $2^{A.size()}$ elements).
            \item $A \subseteq B$ means that A is a subset of B.
            \item $A \subset B$ means that A is a proper subset of B ($A \ne B$).
            \item $A \cup B$ means "A union B", is the set of all elements in (A or B).
            \item $A \cap B$ means "A intersect B", is the set of all elements in (A and B).
            \item $A \times B$ is the cartesian product of A and B, which is the set of all ordered pairs $\{(a, b) : a \in A \wedge b \in B\}$.
            \item $A $
        \end{itemize}

    
\end{itemize}