\subsection{Mathematical Statements}

\begin{mdframed}Investigate!
    \begin{enumerate}
        \item Troll 1 and Troll 3 are knights. Troll 2 is a knave.
    \end{enumerate}
\end{mdframed}

\begin{itemize}
    \item A statement is a sentence that can be true or false.
    
    \item An atomic statement is a statement that cannot be divided into other statements.
    
    \item A molecular statement is a statement that can be divided into other statements.
    
    \item A binary connective is a logical connective that connects 2 statements together. A unary connective is a logical connective that connects 1.
    
    \item Propositional variables are used to represent statements.
    
    \item $\vee$ = OR (inclusive)\newline$\wedge$ = AND\newline$\neg$ = NOT
    
    \item $P \rightarrow Q$ means "P implies Q". This is true only when P is false or P and Q are true. $P$, in this case, is the hypothesis/antecedent, and $Q$ is the conclusion/consequent.
        
    \item To prove an implication, you get to assume P is true in this case because if P is false, the statement is always true.
    
    \item The converse of an implication is $Q \rightarrow P$. This is not logically equivalent to $P \rightarrow Q$.
    
    \item The contrapositive of an implication is $\neg Q \rightarrow \neg P$. This is logically equivalent.
    
    \item If both $P \rightarrow Q$ and $Q \rightarrow P$, then $P \leftrightarrow Q$. This is defined as "if and only if" or "iff".
    
    \item $P \rightarrow Q$ = "P is sufficient for Q"\newline$Q \rightarrow P$ = "P is necessary for Q"\newline$P \leftrightarrow Q$ = "P is necessary and sufficient for Q".
\end{itemize}

\begin{mdframed}Investigate!
    \begin{enumerate}
        \item 1 and 3 are equivalent. 2 and 4 are equivalent.
    \end{enumerate}
\end{mdframed}

\begin{itemize}
    \item A free variable is a variable that is not specified beforehand.
    
    \item A sentence that contains variables is called a predicate.
    
    \item There are two types of quantifiers in mathematics:
    \newline Existential = $\exists x$ = "There exists"
    \newline Universal = $\forall x$ = "For all"
    
    \item Negated quantifer turns into the other one. For example, $\neg\exists x P(x)$ is equivalent to $\forall x \neg P(X)$.
\end{itemize}

\subsubsection{Exercises}
\begin{enumerate}
    \item 
        \begin{enumerate}
            \item Not a statement
            \item Atomic statement
            \item Molecular statement
        \end{enumerate}
    \item
        \begin{enumerate}
            \item Not a statement
            \item Atomic statement
            \item Molecular statement with conditional.
            \item Molecular statement with biconditional.
            \item Molecular statement with disjunction.
            \item Molecular statement with negation.
        \end{enumerate}
    \item
        \begin{enumerate}
            \item $P \wedge Q$
            \item $P \rightarrow \neg Q$
            \item "Jack passed math or Jill passed math."
            \item "If Jack did not pass math or Jill did not pass math, then Jill passed math."
            \item 
                \begin{enumerate}
                    \item Only that Jill passed math.
                    \item Jack did not pass math.
                \end{enumerate}
        \end{enumerate}
    \item
        \begin{enumerate}
            \item Not enough info.
            \item True
            \item True
            \item True
            \item Not enough info.
            \item False
            \item True
        \end{enumerate}
    \item
        \begin{enumerate}
            \item False
            \item Not enough info.
            \item True
            \item Not enough info.
            \item True
        \end{enumerate}
    \item 
        \begin{enumerate}
            \item False
            \item True
            \item False
            \item True
        \end{enumerate}
    \item b is the contrapositive, and c is the converse. 
    \item 
        \begin{enumerate}
            \item If Oscar drinks milk, then he eats Chinese food.
            \item If Oscar does not drink milk, then he does not eat Chinese food.
            \item If the contrapositive was false, then the original statement would be false. They are equivalent.
            \item No. Q is not a part of the if statement, it is only the result.
            \item Yes. You know that since he does not drink milk, he will not eat Chinese food. This is because of the contrapositive.
        \end{enumerate}
    \item 
        \begin{enumerate}
            \item Converse
            \item Original
            \item Converse
            \item Original
            \item Original
            \item Converse
            \item Converse
            \item Original
        \end{enumerate}
\end{enumerate}