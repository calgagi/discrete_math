\subsection{Mathematical Statements}

\begin{mdframed}Investigate!
    \begin{enumerate}
        \item Troll 1 and Troll 3 are knights. Troll 2 is a knave.
    \end{enumerate}
\end{mdframed}

\begin{itemize}
    \item A statement is a sentence that can be true or false.
    
    \item An atomic statement is a statement that cannot be divided into other statements.
    
    \item A molecular statement is a statement that can be divided into other statements.
    
    \item A binary connective is a logical connective that connects 2 statements together. A unary connective is a logical connective that connects 1.
    
    \item Propositional variables are used to represent statements.
    
    \item $\vee$ = OR (inclusive)\newline$\wedge$ = AND\newline$\neg$ = NOT
    
    \item $P \rightarrow Q$ means "P implies Q". This is true only when P is false or P and Q are true. $P$, in this case, is the hypothesis/antecedent, and $Q$ is the conclusion/consequent.
        
    \item To prove an implication, you get to assume P is true in this case because if P is false, the statement is always true.
    
    \item The converse of an implication is $Q \rightarrow P$. This is not logically equivalent to $P \rightarrow Q$.
    
    \item The contrapositive of an implication is $\neg Q \rightarrow \neg P$. This is logically equivalent.
    
    \item If both $P \rightarrow Q$ and $Q \rightarrow P$, then $P \leftrightarrow Q$. This is defined as "if and only if" or "iff".
    
    \item $P \rightarrow Q$ = "P is sufficient for Q"\newline$Q \rightarrow P$ = "P is necessary for Q"\newline$P \leftrightarrow Q$ = "P is necessary and sufficient for Q".
\end{itemize}

\begin{mdframed}Investigate!
    \begin{enumerate}
        \item 1 and 3 are equivalent. 2 and 4 are equivalent.
    \end{enumerate}
\end{mdframed}

\begin{itemize}
    \item A free variable is a variable that is not specified beforehand.
    
    \item A sentence that contains variables is called a predicate.
    
    \item There are two types of quantifiers in mathematics:
    \newline Existential = $\exists x$ = "There exists"
    \newline Universal = $\forall x$ = "For all"
    
    \item Negated quantifer turns into the other one. For example, $\neg\exists x P(x)$ is equivalent to $\forall x \neg P(X)$.
\end{itemize}